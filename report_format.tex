\documentclass[a4j,titlepage]{jarticle}
\usepackage{url}
\usepackage{amsmath,amssymb}
\usepackage[dvipdfmx]{graphicx}

\title{基礎工学PBL 最終レポート}
\author{班名: 8班\\
担当教員: 松下 誠 准教授\\
氏名/学籍番号/電子メールアドレス:\\
岩越 唯人 09B23008 u618941g@ecs.osaka-u.ac.jp\\
梶村 優太 09B23019 u685627d@ecs.osaka-u.ac.jp\\
川原 歩夢 09B23021 u682459c@ecs.osaca-u.ac.jp\\
清原 佑介 09B23026 u643925a@ecs.osaka-u.ac.jp\\
坂井 峻大 09B23033 u681747f@ecs.osaka-u.ac.jp\\
林 宏祐 09B23065 u653793g@ecs.osaka-u.ac.jp
}
\date{提出日:\today}
\begin{document}
\maketitle

\section{課題1,2の概要}
私たち8班は,「高校教員の長時間労働問題と,これに付随する進路指導の質の低下」という社会問題について調査を行った.

この問題の背景として,教員は進路指導を行う義務を負うと法的に定められていること,また教員が生徒を様々な側面において支援するという実務的な問題などが存在している.その結果,高校教員の労働時間が増大する中で進路指導を行わなければならず,進路選択のミスマッチを誘発し得る,進路指導の質の低下が観測されている.また,別の側面では,教職員が長時間労働を強いられていることが,教職員における休職者数や自殺者数の増加に寄与していることも問題視されている.

これに対し,民間・行政ともに様々な対策を練っている.キャリア教育プログラム,職業体験,といった,情報科学技術を強く必要としないものから,求人票のデータ化を支援するサービスや,試験の点数や興味を持っている分野などのデータを蓄積できるシステムなど,情報科学技術を活用したものまで存在する.

以上のことを踏まえ,私たちは,あまり一般的ではない,「生徒が様々な質問に回答することで,希望に沿った大学群が提示される,日常的に利用できるサービス」の開発を目指すことにした.

具体的な方法を説明する.まず,大学の学部・学科,専攻,といったデータを集め,それを1つのデータベースに集約させる.ここに,生徒自身のデータ,例えば定期試験の点数,内申点,希望する専攻分野などを追加する.その上で,生徒にウェブ上で進路決定に関するいくつかの質問に答えてもらう,その結果と生徒の情報を比較して,生徒一人一人に適した条件を持つ,またはその条件に近い大学を提示する.

私たちは,このサービスの開発・普及により,生徒には進路に関する選択肢の適正化や幅の広さを知ってもらうきっかけに,教職員には進路指導に割く時間の削減につながる糸口になることを期待している.しかし,サービス実装に対する残された時間を考慮して,私たちは「生徒に進路の選択肢を提供するシステムのアルゴリズムを考案し,それを実装する」ことを目的と定めた.

\section{採用したソフトウェア,採用した理由}

\section{メンバー間での作業分担方針,実際の作業分担結果}

\subsection{作業分担方針}
\begin{itemize}
\item フロントエンド : 岩越唯人,坂井峻大,林宏祐
\item バックエンド : 梶村優太,清原佑介
\item API通信・全体のサポート : 川原歩夢
\end{itemize}

\subsection{実際の作業分担結果}


\section{動作環境}

\section{環境構築手順}

\section{動作結果}

\section{課題3で最も苦労した点,次に苦労した点}

\section{発表会終了時点でのTODOリスト}
以下,○を未完了事項,●を完了事項とする.\\
\\
●進路指導に必要なパラメータを考える\\
\\
●質問1つからなる簡単な進路指導システムの完成\\
\\
○生徒の偏差値帯を考慮したアプリケーションの作成\\

\section{感想・謝辞}
本授業の遂行にあたり,大阪大学 大学院 情報科学科 松下 誠准教授にはレポートの書き方,効率的なチーム開発の方法等多岐に渡る指導を頂きました.また他の先生方にも議事録を通してのフィードバックや的確なアドバイスを頂き大変勉強になりました.ありがとうございました.
%\begin{thebibliography}[99]
%\item{}
%\end{thebibliography}

\end{document}
